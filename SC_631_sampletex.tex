%% The lines that are not written within % .. % are not to be tampered with!!

\documentclass[twoside]{article}
\setlength{\oddsidemargin}{0.25 in}
\setlength{\evensidemargin}{-0.25 in}
\setlength{\topmargin}{-0.6 in}
\setlength{\textwidth}{6.5 in}
\setlength{\textheight}{8.5 in}
\setlength{\headsep}{0.75 in}
\setlength{\parindent}{0 in}
\setlength{\parskip}{0.1 in}

%
% ADD PACKAGES here:
\usepackage{amssymb}
%

\usepackage{amsmath,amsfonts,graphicx}


\newcounter{lecnum}
\renewcommand{\thepage}{\thelecnum-\arabic{page}}
\renewcommand{\thesection}{\thelecnum.\arabic{section}}
\renewcommand{\theequation}{\thelecnum.\arabic{equation}}
\renewcommand{\thefigure}{\thelecnum.\arabic{figure}}
\renewcommand{\thetable}{\thelecnum.\arabic{table}}


\newcommand{\lecture}[4]{
   \pagestyle{myheadings}
   \thispagestyle{plain}
   \newpage
   \setcounter{lecnum}{#1}
   \setcounter{page}{1}
   \noindent
   \begin{center}
   \framebox{
      \vbox{\vspace{2mm}
    \hbox to 6.28in { {\bf SC 631: Games and Information
		\hfill Autumn 2014} }
       \vspace{4mm}
       \hbox to 6.28in { {\Large \hfill Lecture #1: #2  \hfill} }
       \vspace{2mm}
       \hbox to 6.28in { {\it Instructor: #3 \hfill Scribes: #4} }
      \vspace{2mm}}
   }
   \end{center}
   \markboth{Lecture #1: #2}{Lecture #1: #2}

   {\bf Note}: {\it LaTeX template courtesy of UC Berkeley EECS dept.}

   {\bf Disclaimer}: {\it These notes have not been subjected to the
   usual scrutiny reserved for formal publications.  They may be distributed
   outside this class only with the permission of the Instructor.}
   \vspace*{4mm}
}



\renewcommand{\cite}[1]{[#1]}
\def\beginrefs{\begin{list}%
        {[\arabic{equation}]}{\usecounter{equation}
         \setlength{\leftmargin}{2.0truecm}\setlength{\labelsep}{0.4truecm}%
         \setlength{\labelwidth}{1.6truecm}}}
\def\endrefs{\end{list}}
\def\bibentry#1{\item[\hbox{[#1]}]}

%Use this command for a figure; it puts a figure in wherever you want it.
%usage: \fig{NUMBER}{SPACE-IN-INCHES}{CAPTION}
%This is for your help while typing the notes
\newcommand{\fig}[3]{
			\vspace{#2}
			\begin{center}
			Figure \thelecnum.#1:~#3
			\end{center}
	}
% Use these for theorems, lemmas, proofs, etc. 
% See how they are used, below 
\newtheorem{theorem}{Theorem}[lecnum]
\newtheorem{lemma}[theorem]{Lemma}

\newcommand\E{\mathbb{E}}
\newcommand{\PI}{$P_{i}$}
\newcommand{\KIA}{$K_{i}(A)$}
\newcommand{\FIW}{$F_{i}(\omega)$}
\newcommand{\OS}{$\omega ^{*}$}
\begin{document}

\lecture{1}{August 14}{Ankur A. Kulkarni}{Tarun, Mridul and Deependra}%In place of scribe-name, write down your names or your group names

This lecture's notes illustrate some uses of
various \LaTeX\ macros.  
Take a look at this and imitate.
\section{Some Definitions}
\subsection{Event}
An event is a subset of Y if w=Y is the state of the world then we say event A obtains in w, if w=A.
\subsection{$F_{i}(\omega)$}
$F_{i}(\omega) \triangleq$ that event F of Fi s.t. $\omega \in F$
\subsection{Knowing an Event}
$P_{i}$ knows event A in state of the world $\omega$ if $F_{i}(\omega) \subseteq A$ \\
$\implies \; \; \forall \; \omega \prime \in F_{i}(\omega) \implies \omega \prime \in A$
\subsection{Knowledge Operator \KIA}
\KIA = \{ $\omega \in Y \; | \; F_{i}(A)\subseteq A$ \}
\KIA is the set of all those states of the world when player \PI knows A
\subsection{Common Knowledge}
An Event A is common knowledge in state of the world $\omega$ if for all $i_{1}, i_{2}, .....i_{n} \in \mathit{N}, n< \infty $ \\
$\omega\in \; K_{i_{1}}K_{i_{2}}K_{i_{3}}.....K_{i_{n}}(A)$
\section{Some theorems and stuff}
\begin{itemize}
\item If \OS $\in$ Y is the state of the world then \PI \; knows A in \OS \; iff \OS $\in$ \KIA
\item K$_{j}(K_{i}(A))$ = Event that P$_{j}$ knows that \PI knows A
\end{itemize}
\section{Important Properties}
\begin{itemize}
\item \KIA $\subseteq$ A
\item A $\subseteq$ B \\
	$\implies$ \KIA $\subseteq$ K$_{i}(B)$ 
\item $\forall A\subseteq$ Y, $K_{i}K_{i}(A)$ = \KIA
\item \KIA $\cap K_{i}(B)$ = $K_{i}(A \cap B)$
\item $K_{i}($(\KIA $)^{c})$ = (\KIA)$^{c}$
\end{itemize}
\begin{lemma}
If A is common knowledge in $\omega$, and B$\supseteq$A, then \\
B is also common knowledge in $\omega$(follows from property 2)
\end{lemma}
\begin{lemma}
If A is common knowledge in $\omega$, then\\
$\omega \in$ \KIA and \FIW $\subseteq$ A for i
\end{lemma}
\begin{theorem}
If A is common knowledge in $\omega$ and $\omega\prime\in$\FIW \; for some i $\in\mathit{N}$, then
A is common knowledge in $\omega\prime$
\end{theorem}
A is common knowledge in $\omega\\\implies$
$\omega\in \; K_{i}K_{i_{1}}K_{i_{2}}.....K_{i_{n}}(A) \; \forall i_{1}, i_{2}, .....i_{n} \in \mathit{N}$\\
Suppose $\omega\prime\in \; $\FIW, \\
$\omega \in K_{i}(K_{i_{1}}K_{i_{2}}.....K_{i_{n}}(A))$ \\
$\implies$ \FIW $\subseteq K_{i_{1}}K_{i_{2}}.....K_{i_{n}}(A)$ \\
$\implies \omega\prime\in K_{i_{1}}K_{i_{2}}.....K_{i_{n}}(A)$
%% THIS ENDS THE EXAMPLES. DON'T DELETE THE FOLLOWING LINE:
\section{Next topic}

Here is a citation, just for fun~\cite{CW87}.

\section*{References}
\beginrefs
\bibentry{CW87}{\sc Author1} and {\sc Author2}, 
Title of the paper
{\it Name of the Journal},
2014, pp.~1--6.
\endrefs
\end{document}